\documentclass[12pt,letterpaper]{article}


\usepackage[T1]{fontenc}
\usepackage[utf8]{inputenc}
\usepackage{times}
\usepackage[left=3.5cm,top=2.5cm,bottom=2.5mm,right=3.5cm]{geometry}

\begin{document}


\title{Práctica 1: EV 1 1 Circuitos de reactificación no controlados.\\ Universidad Politécnica de la Zona Metropolitana de Guadalajara.\\ 
Ing Mecatrónica. Maestro: Ing. Carlos Enrique Morán Garabito.}\\
\author{Nombres: Capuchino González Jonathan Alejandro, Fernández Gaeta Uriel y Salcedo González Alondra}}

\section{Objetivo}
Aprender las diferentes ondas que se dan en los diferentes circuitos.

\section{Materiales}
Solamente se utilizó la computadora que anteriormente tenga instalado el programa de simulación a nivel industrial Orcad.

\section{Procedimiento}
El maestro nos proporcionó un documento de 40 hojas en la cual contiene una serie de circuitos que teníamos que simular para ver las diferentes ondas que dá los diodos.
En total fueron 7 circuitos en el cual, uno de ellos se tenía que hacer una modificación para poder utlizarlo dentro de un circuito cómo si fuera un componente.
Los resultados fueron los siguientes:\\ 
Los circuitos 1, 6, 7 cuando se hicieron las simulaciones se daban las ondas completas.\\
Los circuitos 2, 3, 4, 5 cuando se hicieron las simulaciones se daban las medias ondas.\\
Támbien se presentó diferentes configuraciones para también mostrar la armonía, frecuencia en Hz en forma de tablas.

\section{Conclusión}
Alondra: El mayor problema fue que como ser un nuevo programa se nos hizo un poco enredoso ya que no es lo mismo que utilizar proteus o multisim que son los pragramas de simulación más comúnes que se llegan a utilizar.\\
También en el circuito 5 en las hojas tienen las indicaciones de hacer el subcircuito, el mayor problema que se presentó fue que hubo complicaciones para guardar el subcircuito cómo un símbolo, el objetivo de guardarlo de esta manera era para hacerlo como un componente pero al final no se pudo, así que decidimos ponerle fuente y ver las ondas que nos daban como una señal.

\end{document}